\documentclass{article}
\usepackage[utf8]{inputenc}
\usepackage{longtable}
\usepackage{geometry}
\geometry{a4paper, left=0in, right=0in, top=0in, bottom=0in}

\begin{document}

\begin{longtable}{|c|l|p{10cm}|}
\hline
\textbf{Código EAP} & \textbf{Atividade} & \textbf{Explicação} \\ \hline
1 & Desenvolvimento da Logística do Almoço & Planejamento, desenvolvimento, testes e implantação do sistema de logística para o almoço escolar. \\ \hline
1.1 & Planejamento & Fase inicial de organização e definição do projeto. \\ \hline
1.1.1 & Levantamento de Requisitos & Identificar e documentar todas as necessidades funcionais e não funcionais do sistema. \\ \hline
1.1.2 & Análise de Viabilidade & Avaliar a viabilidade técnica e econômica do projeto. \\ \hline
1.1.3 & Definição do Escopo & Estabelecer os limites e objetivos específicos do sistema. \\ \hline
1.1.4 & Definição do Cronograma & Criar um cronograma detalhado para as etapas do projeto. \\ \hline
1.1.5 & Contratação e Seleção & Selecionar fornecedores e contratar os profissionais necessários. \\ \hline
1.2 & Desenvolvimento & Fase de criação e implementação do sistema. \\ \hline
1.2.1 & Arquitetura e Design & Projetar a estrutura técnica e visual do sistema. \\ \hline
1.2.2 & Cadastro dos Alunos no Sistema & Registrar os dados dos alunos na plataforma. \\ \hline
1.2.3 & Registro do Saldo do Cartão & Implementar funcionalidade para gerenciar o saldo dos cartões dos alunos. \\ \hline
1.2.4 & Cadastro do Histórico de Compra dos Alunos & Criar o registro detalhado de compras realizadas pelos alunos. \\ \hline
1.2.5 & Canal de Comunicação Aluno-Suporte & Desenvolver um meio de comunicação entre alunos e suporte técnico. \\ \hline
1.3 & Testes & Validar o funcionamento e a qualidade do sistema. \\ \hline
1.3.1 & Testes de Integração & Garantir que os diferentes módulos funcionem corretamente em conjunto. \\ \hline
1.3.2 & Testes de Usabilidade & Avaliar a experiência do usuário no sistema. \\ \hline
1.3.3 & Testes de Segurança & Identificar e corrigir vulnerabilidades no sistema. \\ \hline
1.3.4 & Testes de Performance & Testar o desempenho do sistema sob diferentes cargas de trabalho. \\ \hline
1.4 & Implantação & Publicar o sistema e torná-lo operacional. \\ \hline
1.4.1 & Hospedagem em Produção & Disponibilizar o sistema no ambiente de produção. \\ \hline
1.4.2 & Configuração do Banco de Dados & Configurar e otimizar o banco de dados para uso em produção. \\ \hline
1.4.3 & Configuração do Backup & Implementar mecanismos de backup para proteger os dados. \\ \hline
1.4.4 & Disponibilização para o Usuário & Tornar o sistema acessível aos usuários finais. \\ \hline
1.5 & Treinamento e Suporte & Capacitar os usuários e oferecer suporte técnico contínuo. \\ \hline
1.5.1 & Treinamento da Equipe de Suporte & Capacitar a equipe para atendimento técnico, uso de ferramentas e protocolos de suporte. \\ \hline
1.5.2 & Treinamento dos Usuários Finais & Instruir os usuários sobre o uso eficiente e correto do sistema. \\ \hline
1.5.3 & Desenvolvimento de Materiais de Suporte & Criar materiais como manuais, FAQs e vídeos para facilitar o uso do sistema. \\ \hline
1.5.4 & Suporte Técnico Pós-Lançamento & Resolver problemas e ajustar o sistema após a implantação. \\ \hline
1.5.5 & Acompanhamento e Ajustes Pós-Implantação & Monitorar o sistema e realizar ajustes baseados no feedback dos usuários. \\ \hline
\caption{Estrutura Analítica do Projeto (EAP) - Desenvolvimento da Logística do Almoço} \label{tab:eap-logistica-almoco} \\
\end{longtable}

\end{document}